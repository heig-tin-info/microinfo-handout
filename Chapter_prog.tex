\chapter{\textsc{Programmation en C}}
Pour atteindre le mat�riel depuis un langage de plus haut niveau, il faut un certain nombre d'instructions et de r�gles pour pouvoir rester dans un ensemble coh�rent.


\section{Programmation syst�me et programmation applicative}
Les r�gles de programmation sont diff�rentes si on �crit du code relatif � du mat�riel par rapport � des applications purement logiciel. Le premier exemple, d�j� trait� au chapitre 4, est les interruptions. Ce sont des fonctions syst�me que l'on trouve seulement lorsqu'on programme � bas niveau. 

\subsection{Directive pragma}
La directive \#pragma est d�finie par le standard ISO/ANSI C. Elle permet de garder une certaine portabilit� du logiciel en contr�lant de mani�re particuli�re les extensions sp�cifiques � un fabricant. Cette directive permet de  sp�cifier une adresse absolue qui d�signe la position m�moire pour la d�claration suivant cette directive. La variable doit �tre d�clar�e soit \_\_no\_init ou const. 

\lstset{style=customc}
\begin{lstlisting}
#pragma location=0x22E
__no_init volatile char PORT1;	//PORT1 est un registre du p�riph�rique
                                //du m�me nom plac� � l'adresse 0x22E
\end{lstlisting}



\section{Librairies d'abstraction du mat�riel: HAL}
Pour rendre l'�criture de programmes plus propre on peut ajouter de la hi�rarchie. Une fa�on de faire est de placer toutes les fonctions relatives aux p�riph�riques dans des librairies d'abstraction de mat�riel ou HAL (Hardware Abstraction Library).




