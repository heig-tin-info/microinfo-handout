\chapter{\textsc{Programmation en C}}
Pour atteindre le matériel depuis un langage de plus haut niveau, il faut un certain nombre d'instructions et de règles pour pouvoir rester dans un ensemble cohérent.


\section{Programmation système et programmation applicative}
Les règles de programmation sont différentes si on écrit du code relatif à du matériel par rapport à des applications purement logiciel. Le premier exemple, déjà traité au chapitre 4, est les interruptions. Ce sont des fonctions système que l'on trouve seulement lorsqu'on programme à bas niveau. 

\begin{table}[!htbp]
\begin{center}
{\fontfamily{phv}\selectfont
\begin{tabular}{|c|c|}
\hline 
Programmation Système & Programmation Applicative \\
\hline  
\hline 
Interruptions & Pas d'interruptions \\
\hline 
Variables globales & Éviter les variables globales\\
\hline
Langages de bas niveau & Langages de haut niveau\\
\hline
Structuration faible & Structuration forte\\
\hline
\end{tabular}
}
\end{center}
\caption{Comparaison des styles de programmation \label{prog}}
\end{table}

\section{Directives et variables spécifiques à la programmation système}

\subsection{Variables volatiles}
Dans le cas ou une zone de stockage, tel qu'un registre, peut être modifié par autre chose que le programme principal on doit lui attribuer le modificateur volatile. Cela permet au compilateur de savoir que cette variable peut changer même si le code n'affecte jamais cette variable. Le compilateur ne va donc pas essayer d'optimiser une variable volatile (Exemple ci-dessous). 

\subsection{Directive pragma}
La directive \#pragma est une commande de précompilation définie par le standard ISO/ANSI C. Elle permet de garder une certaine portabilité du logiciel en contrôlant de manière particulière les extensions spécifiques à un fabricant. Cette directive permet, par exemple, de spécifier une adresse absolue qui désigne la position mémoire pour la déclaration suivante. La variable doit être déclarée soit \_\_no\_init ou const. 

\lstset{style=customc}
\begin{lstlisting}
#pragma location=0x22E
__no_init volatile char PORT1;	//PORT1 est un registre du périphérique
                                //du même nom placé à l'adresse 0x22E
\end{lstlisting}
La directive \#pragma peut prendre d'autres fonctions selon le compilateur utilisé.

\subsection{Fonctions intrinsèques}
Les fonctions intrinsèques se reconnaissent par un double souligné devant (\_\_). Les fonctions intrinsèques permettent un accès direct aux instructions de bas niveau sans programmer en assembleur. Si le compilateur ne trouve pas les fonctions intrinsèques par défaut alors il faut inclure: intrinsics.h. Ces fonctions ne sont pas standard, elle dépendent du compilateur utilisé.

Exemple de fonctions intrinsèques:
\lstset{style=customc}
\begin{lstlisting}
__no_init					//indique de ne pas initialiser une variable
__disable_interrupt();		//désactive toutes les interruptions
__enable_interrupt();		//active toutes les interruptions
__low_power_mode_n()			//enclenche le mode low power [0..4]
__low_power_mode_off_on_exit() //désactive le mode low power en sortie d'interruption
__asm()						//permet l'insertion de code assembleur
\end{lstlisting}


\section{Librairies d'abstraction du matériel: HAL}
Pour rendre l'écriture de programmes plus propre on peut ajouter de la hiérarchie. Une façon de faire est de placer toutes les fonctions relatives aux périphériques dans des librairies d'abstraction de matériel ou HAL (Hardware Abstraction Library).

Une librairie HAL doit permettre de masquer tout ce qui est relatif à un périphérique donné. La structure typique d'un HAL est la suivante:
\begin{itemize}
\item Constantes spécifiques
\item Variables spécifiques
\item Une fonction d'initialisation du périphérique
\item Une fonction de mise en veille ou désactivation
\item Des fonctions spécifiques comme lecture ou écriture
\end{itemize}

Exemples de structure HAL pour un accéléromètre (CMA3000):
\lstset{style=customc}
\begin{lstlisting}
#define CTRL              0x02
#define MODE_400          0x04        // Measurement mode 400 Hz ODR

int8_t Cma3000_xAccel;
int8_t Cma3000_yAccel;
int8_t Cma3000_zAccel;

void Cma3000_init(void)
{
	...
}

void Cma3000_disable(void)
{
	...
}

void Cma3000_readAccel(void)
{
	...
} 

...
\end{lstlisting}

Les interruptions relatives à un périphérique doivent être aussi traitées dans le HAL.

\section{Pilotes}
Un pilote ou driver permet la gestion d'un périphérique dans le cadre d'un système d'exploitation. C'est un HAL avec en plus les interfaces standards propre à un système d'exploitation. Il existe donc des standards pour écrire des pilotes. 

\section{Contrôle de l'exécution}

Dans le cadre d'une programmation simple (mono-tâche), le processeur effectue les instructions linéairement à sa vitesse maximale. On a donc des instants actifs, avec des instructions à exécuter, et des instants inactifs où il n'y a rien à faire. Dans le cas le plus simple, la partie inactive est une attente active ou le processeur exécute une boucle infinie genre: while(1). Ceci revient à faire le même test qui est toujours vrai et donc à gaspiller de l'énergie. Il est de ce fait plus intéressant de mettre le processeur en pause. Ceci implique qu'il faut gérer le réveil au moment requis par le système.

Le cycle actif/pause peut être géré de deux manières:
\begin{enumerate}
\item Par interruption, une requête externe par exemple
\item Par un timer, réveil à intervalle régulier.
\end{enumerate}

Exemple de code pour MSP430 dans le cas (1):
\lstset{style=customc}
\begin{lstlisting}
void main(void)
{
	WDTCTL = WDTPW | WDTHOLD;              // Stop watchdog timer

	while(1) {
		__bis_SR_register(LPM3_bits + GIE);	//Mode basse consommation LPM3 
											//avec interruptions activées
		...									//Code de l'application, 
											//mode low power!
	}
}

#pragma vector=PORT2_VECTOR
__interrupt void port2_isr(void) {

	...										//Code de l'interruption, 
											//mode actif par défaut
	__low_power_mode_off_on_exit();	
}
\end{lstlisting}

Pour bien comprendre le code ci-dessus, il faut savoir qu'une interruption est toujours exécutée en mode actif (pas low power) et que dès la sortie, le processeur retourne dans le mode d'avant l'interruption. Il faut donc utiliser la fonction intrinsèque \_\_low\_power\_mode\_off\_on\_exit pour retourner dans le programme principal en mode actif et exécuter le code de l'application. Le mécanisme qui régit cette opération est la sauvegarde du contexte qui sauve sur la pile le registre de statut SR lors de l'interruption puis le dépile lors de la sortie. Rappel: les modes basse consommation sont désignés dans le SR.

Exemple de code pour MSP430 dans le cas (2):
\lstset{style=customc}
\begin{lstlisting}
void main(void)
{
	WDTCTL = WDTPW | WDTHOLD;              // Stop watchdog timer

	while(1) {		
		__bis_SR_register(LPM3_bits + GIE);	//Mode basse consommation LPM3 
											//avec interruptions activées
		...									//Code de l'application, 
											//mode low power!
	}
}

#pragma vector=TIMER_A0_VECTOR
__interrupt void timer_a0_isr(void)
{
	...										//Code de l'interruption, 
											//mode actif par défaut
	__low_power_mode_off_on_exit();	
}
\end{lstlisting}
