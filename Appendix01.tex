\chapter{\textsc{Démonstrations algébriques}}
\label{ch:algebra}

De manière générale, on peut écrire un nombre entier non-signé sous la forme suivante:
\begin{equation}
\begin{aligned}
A_{10} &= \sum_{i=0}^{n-1} a_{i}2^{i} = a_{n-1}2^{n-1} + a_{n-2}2^{n-2} +...+ a_{1}2^{1} + a_{0}2^{0} \\
avec &:  a_{i} \in \{0,1\} \\
et &:   A_{10} \in [0...2^{n} - 1]
\end{aligned}
\end{equation}

Un nombre entier signé, en complément à deux, peut être écrit sous cette forme:
\begin{equation}
A_{10} = -a_{n-1}2^{n-1} + \sum_{i=0}^{n-2} a_{i}2^{i} = -a_{n-1}2^{n-1} + a_{n-2}2^{n-2} +...+ a_{1}2^{1} + a_{0}2^{0}
\end{equation}

\section{Changement de signe}

\begin{equation}
\begin{aligned}
-A_{10} &= \enspace - \Bigg( \underbrace{ -a_{n-1}2^{n-1} + \sum_{i=0}^{n-2} a_{i}2^{i}}_{A_{10}} \Bigg) \enspace = \enspace \underbrace{-2^{n-1} + \underbrace{\sum_{i=0}^{n-2} 2^{i} + 1}_{2^{n-1}}}_{0} + a_{n-1}2^{n-1} - \sum_{i=0}^{n-2} a_{i}2^{i} \\ &= \enspace -(1-a_{n-1})2^{n-1} + \sum_{i=0}^{n-2}(1-a_i)2^i + 1 \enspace = \enspace -\bar{a}_{n-1}2^{n-1} + \sum_{i=0}^{n-2} \bar{a}_{i}2^{i} + 1 \\ &= \enspace \bar{A} + 1
\end{aligned}
\end{equation}

\section{Extension d'un nombre signé}

On cherche à trouver les bits $b_i$ qui permettent d'étendre un nombre négatif ($a_{n-1}=1$) en complément à deux de n bits à m bits:
\[ a_{n-1} a_{n-2} ... a_{0} \Rightarrow b_{m-1} b_{m-2} ... b_{n-1} a_{n-2} ... a_{0} \]

Exemple:
\[m =8, n=4: a_3 a_2 a_1 a_0 = b_7 b_6 b_5 b_4 b_3 a_2 a_1 a_0 \]
Avec $a_3$ le bit de signe original et $b_7$ le bit de signe du nombre étendu. $a_3$ se déplace donc en première position $b_7$ pour conserver le bit de signe. Comme ils valent 1 on les simplifies et la notation peut s'écrire de manière générale:

\begin{equation}
\begin{aligned}
-2^{n-1} + \sum_{i=0}^{n-2}a_i 2^i \enspace &= \enspace -2^{m-1} + \sum_{i=n-1}^{m-2} b_{i}2^{i} + \sum_{i=0}^{n-2}a_i 2^i  \quad m > n\\
-2^{n-1} \enspace &= \enspace -2^{m-1} + \sum_{i=n-1}^{m-2} b_{i}2^{i} \\
2^{m-1} \enspace &= \enspace 2^{n-1} + \sum_{i=n-1}^{m-2} b_{i}2^{i}
\end{aligned}
\end{equation}

Cette égalité est toujours vraie si est seulement si les $b_i$ sont égaux à 1 et quelque soit m, n (m > n). Exemples:
\[
m=4, n=2: \quad 2^{4-1} = 2^{2-1} + b_1 2^{1} + b_2 2^{2} \Leftrightarrow 8 = 2 + 1 \cdot 2 + 1 \cdot 4
\]

\[
m=8, n=4: \quad 2^{8-1} = 2^{4-1} + b_3 2^{3} + b_4 2^{4} + b_5 2^{5} + b_6 2^{6}\Leftrightarrow 128 = 8 + 1 \cdot 8 + 1 \cdot 16 + 1 \cdot 32 + 1 \cdot 64
\]

\section{Multiplication de nombres en complément à deux}
La multiplication en colonnes classique fonctionne en complément à 2 mais elle nécessite une correction à la fin du calcul en fonction du signe des opérandes pour retrouver le résultat correct. Soit m et r, deux entier sur n bits:

\begin{equation}
\begin{aligned}
+m \equiv m \enspace & \enspace +r \equiv r\\
-m \equiv 2^n-m \enspace & \enspace -r \equiv 2^n-r\\
(+m) \times (+r) \enspace &= \enspace mr\\
(-m) \times (+r) \enspace &= \enspace 2^{n}r - mr\\
(+m) \times (-r) \enspace &= \enspace 2^{n}m - mr\\
(-m) \times (-r) \enspace &= \enspace 2^{n}2^{n} - 2^{n}m - 2^{n}r + rm
\end{aligned}
\end{equation}

Le facteur $2^{n}2^{n}$ disparait par arithmétique modulo. Il reste néanmoins une correction à effectuer qui est différente pour chaque cas. Cette différence de temps de calcul en fonction du signe est gênante mais peut être évitée grâce à l'algorithme de Booth (1950) \cite{booth}. L'algorithme consiste en la transformation suivante:

\begin{equation}
\begin{aligned}
A \cdot B &= \enspace B \Bigg( \underbrace{ -a_{n-1}2^{n-1} + \sum_{i=0}^{n-2} a_{i}2^{i}}_{A_{10}} \Bigg) \enspace 
= \enspace  B \Bigg( -a_{n-1}2^{n-1} + 2\sum_{i=0}^{n-2} a_{i}2^{i} - \sum_{i=0}^{n-2} a_{i}2^{i} \Bigg) \\ 
&= \enspace B \Bigg( -a_{n-1}2^{n-1} + \sum_{i=0}^{n-2} a_{i}2^{i+1} - \sum_{i=0}^{n-2} a_{i}2^{i} \Bigg) \enspace 
= \enspace B \Bigg( \sum_{i=1}^{n-1} a_{i-1}2^{i} - \sum_{i=0}^{n-1} a_{i}2^{i} \Bigg) \\ 
&= \enspace B \Bigg( \sum_{i=1}^{n-1} (a_{i-1}-a_i)2^{i} - a_{0}2^{0} \Bigg)
\end{aligned}
\end{equation}

Et comme le terme $a_{-1}$ peut être considéré nul car en dehors du nombre alors:

\begin{equation}
A \cdot B = \enspace \sum_{i=0}^{n-1} B(a_{i-1}-a_i)2^{i}
\end{equation}

On peut montrer numériquement que le résultat de la multiplication est correct indépendamment du signe de A et de B \cite{booth}.

L'algorithme de Booth peut être étendu pour inclure des séquences de "1" grâce à la même propriété utilisée ci-dessus qui est que tout "1" logique peut s'écrire $a_i2^i = a_i2^{i+1}-a_i2^i$ et donc toute séquence de "1" peut s'écrire $a_i2^{i+k}-a_i2^i$ (avec k la longueur de la séquence de 1). Ceci permet de réduire le nombre de multiplications à effectuer. Prenons un exemple pour bien comprendre:
\[
A \cdot B = A \cdot "00111110" = A \cdot (2^5 + 2^4 + 2^3 + 2^2 + 2^1) = A \cdot 62
\]
\[
A \cdot B = A \cdot "00111110" = A \cdot (2^6 - 2^1) = A \cdot (64 - 2)
\]
On passe donc de 5 à 2 termes à calculer.

La réalisation de cet algorithme est simplifiée si on parcours le multiplicateur de gauche à droite en faisant une addition du multiplicande lorsqu'on rencontre "01" (début de séquence) et une soustraction du multiplicande lorsqu'on rencontre "10" (fin de séquence) avec les décalages appropriés. On peut aussi le faire dans l'autre sens avec début à "10" et fin à "01".
Cet algorithme peut être encore étendu en utilisant le concept du codage redondant. Par exemple pour le cas ci-dessus on considère deux bits à chaque itération. On peut aussi en considérer trois ou même quatre à chaque itération ce qui donne plus de flexibilité pour coder avantageusement le multiplicateur.

\begin{thebibliography}{1}

\bibitem[1]{booth}Andrew D. Booth, \emph{A Signed Binary Multiplication Technique}, Journal of Mechanical and Applied Mathematics, 1951
%
%\bibitem[2]{proakis}John G. Proakis, Dimitris G. Manolakis: \emph{Digital Signal Processing}, Pearson Prentice Hall, 2007
%
\end{thebibliography}