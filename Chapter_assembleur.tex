\chapter{\textsc{Assembleur}}

On appelle \textit{langage machine} le langage natif du CPU, c'est � dire le langage que celui-ci \textit{comprend}. Les instructions sont des mots binaires organis�s en champs, qui commandent chacun des actions sp�cifiques telles que :
\begin{itemize}[label=\textbullet,font=\small]
\item le type d'op�ration � effectuer
\item les op�randes
\item le mode d'adressage
\end{itemize}

Un langage d'assemblage (ou langage assembleur ou simplement assembleur par abus de
langage, abr�g� ASM) est le langage de bas niveau proche du langage machine qui peut �tre
directement interpr�t� par le processeur tout en restant lisible par un utilisateur.

Il consiste � repr�senter les codes binaires des instructions par des symboles appel�s mn�moniques (du grec mn�monikos, relatif � la m�moire), c'est-�-dire faciles � retenir.

Par exemple, l?unit� de contr�le d?un processeur particulier reconna�t l?instruction en langage
machine suivante :

          en hexad�cimal : 4035 0055 ou en binaire : 01000000 00110101 00000000 01010101

En langage assembleur, cette instruction sera traduite par un �quivalent plus facile � comprendre
pour le programmeur 

          MOV.W   #85H,R5

Ce qui signifie "mettre la valeur d�cimale 85 (0x55 en hexad�cimal) dans le registre R5.


On appelle \textit{assembleur} le langage natif du CPU, c'est � dire le langage que celui-ci \textit{comprend}. 


\section{Chaine de compilation}


\section{Debug}