\chapter{\textsc{Assembleur}}

TBD


\section{Chaine de compilation}


\section{Debug}